\documentclass{article}
\usepackage[slantfont, boldfont]{xeCJK}
\usepackage{listings}
% 设置英文字体
\setmainfont{Microsoft YaHei}
\setsansfont{Comic Sans MS}
\setmonofont{Courier New}

% 设置中文字体
\setCJKmainfont{Microsoft YaHei}
\setCJKmonofont{Source Code Pro}
\setCJKsansfont{YouYuan}
% 这里是导言区
\title{docker和virtualenv的区别}

%\tiny
%\scriptsize
%\footnotesize
%\small
%\normalsize
%\large
%\Large
%\LARGE
%\huge
%\Huge

\begin{document}
\maketitle
virtualenv只是针对python的隔离工具,是借助语言层面的虚拟机来实现的,\large{不具有通用性}。

\mbox{}

\normalsize Docker是更底层的虚拟化技术,是进程级的,也就是说\large{不管什么语言}\normalsize{编写的程序}\large{都会}\normalsize{得到相应的}\large{服务}。
\normalsize 虚拟化技术分为\large{硬件级}\normalsize{虚拟化和}\large{操作系统}\normalsize{级虚拟化}。
\begin{itemize}
    \item 硬件级:模拟的就是一个操作系统。
    \item 操作系统级:模拟的是运行在操作系统上的多个不同进程。并将其封装在一个密闭的容器里面,因此也称为容器化技术。
\end{itemize}
\end{document}