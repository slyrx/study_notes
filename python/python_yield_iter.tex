\documentclass{article}
\usepackage[slantfont, boldfont]{xeCJK}
\usepackage{listings}
% 设置英文字体
\setmainfont{Microsoft YaHei}
\setsansfont{Comic Sans MS}
\setmonofont{Courier New}

% 设置中文字体
\setCJKmainfont{Microsoft YaHei}
\setCJKmonofont{Source Code Pro}
\setCJKsansfont{YouYuan}
% 这里是导言区
\title{yield和iter的关系}
\begin{document}
\maketitle
\section{iter定义}
{内建函数iter()可以生成一个iterator迭代器。
\begin{itemize}
    \item {优点
    
    
    相比list来说,iterator不需要很大的内存空间。
    
    
    使用迭代器解决了空间占用的问题,不过代码也太繁琐了,一点没有python风格。}
    \item {操作
    
    
    迭代器通过next()来遍历元素,并且完成遍历时抛出\emph{StopIteration()}异常。}
\end{itemize}}
\begin{lstlisting}
    it = iter(range(5))
    print it.next()         # 0
    print it.next()         # 1
    print it.next()         # 2
    print it.next()         # 3
    print it.next()         # 4

    print it.next()         # StopIteration()
\end{lstlisting}

\mbox{}

{用for循环对迭代器进行遍历
\begin{lstlisting}
    it = iter(range(5))
    for i in it:
        print i

    print it.next()         # StopIteration()
\end{lstlisting}}

\mbox{}

遍历完成时,调用it.next(),抛出StopIteration()异常。可以看出for循环调用的是next()方法。就像下边这样。
\begin{lstlisting}
    while True:
    try:
        print it.next()
    except StopIteration:
        break    
\end{lstlisting}

\section{yield定义}
yield是为了应对iter代码太复杂的简化方案,是一个可迭代的并且简洁的方案。
\begin{itemize}
    \item {特点:
    
    
    使用next()方法会依次返回元素,并且越界时报StopIteration异常。}
\end{itemize}
\begin{lstlisting}
def myEnumerate(seq, start=0):
    n = start
    for i in seq:
        yield n, i
        n += 1
\end{lstlisting}
\begin{lstlisting}
it = myEnumerate(range(5))
print it.next()         # (0, 0)
print it.next()         # (1, 1)
print it.next()         # (2, 2)
print it.next()         # (3, 3)
print it.next()         # (4, 4)
print it.next()         # StopIteration    
\end{lstlisting}

\mbox{}

我的理解,yield并没有简化原来写在next里面的代码,而是告诉了next函数在执行的时候,上次存储的位置在哪里,需要从什么地方开始。
\begin{lstlisting}
    def next(self):
        if self.n == len(self.seq):
            raise StopIteration()
        item = self.seq[self.n]
        index = self.start
        self.n += 1
        self.start += 1
        return index, item    
\end{lstlisting}

另外,因为生成器只能被迭代一次。不可以再次使用 for i in mygenerator 。对比普通的[]数组,则会一直保留着这个变量,占用内存空间。
\begin{lstlisting}
第一遍,会输出0,1,4
>>> mygenerator = (x*x for x in range(3))
>>> for i in mygenerator :
...    print(i)
0
1
4
第二遍,什么也不会输出
>>> mygenerator = (x*x for x in range(3))
>>> for i in mygenerator :
...    print(i)

\end{lstlisting}

每次在yield处都要确认是不是这次调用的结尾了,
\begin{itemize}
    \item 如果是结尾,就执行后续的内容。
    \item 如果不是结尾,就再次执行yield以前的内容。
\end{itemize}


yield只执行一次,能重复调用是因为再函数内有循环来调用yield。


yield就像一个断点,当调用一次对应的函数时,执行到这个断点处,并保存一个快照,当再次调用这个函数时,则从断点的位置开始向后执行。

\end{document}